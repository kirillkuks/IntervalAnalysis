\documentclass[a4paper,12pt]{article}

\usepackage[hidelinks]{hyperref}
\usepackage{amsmath}
\usepackage{mathtools}
\usepackage{shorttoc}
\usepackage{cmap}
\usepackage[T2A]{fontenc}
\usepackage[utf8]{inputenc}
\usepackage[english, russian]{babel}
\usepackage{xcolor}
\usepackage{graphicx}
\usepackage{float}
\graphicspath{{./images/}}

\definecolor{linkcolor}{HTML}{000000}
\definecolor{urlcolor}{HTML}{0085FF}
\hypersetup{pdfstartview=FitH,  linkcolor=linkcolor,urlcolor=urlcolor, colorlinks=true}

\DeclarePairedDelimiter{\floor}{\lfloor}{\rfloor}

\renewcommand*\contentsname{Содержание}

\newcommand{\plot}[3]{
    \begin{figure}[H]
        \includegraphics[scale=0.7]{#1}
        \caption{#2}
        \label{#3}
    \end{figure}
}

\begin{document}
    \include{Title}
    \newpage

    \tableofcontents
    \newpage

    \section{Постановка задачи}
    \subsection{Линейный случай}
    Выбрать ИСЛАУ $ 2 \times 2 $ вида:
    \begin{equation}
        \begin{cases}
            a \cdot x_{1} + b \cdot x_{2} = c \\
            1 \cdot x_{1} - k \cdot x_{2} = 0
        \end{cases}
        \label{eq:linear}
    \end{equation}
    где $ a, b $ - положительные числа, $ c, k $ - положительные интервалы.\newline
    Оценить внешнее множество решений этой системы методом Кравчика.
    \begin{itemize}
        \item Определить спектральный радиус матрицы
        \item Провести оценку начального бруса решений
    \end{itemize}
    Провести вычисления и привести иллюстрации:
    \begin{itemize}
        \item Положения брусов при итерациях
        \item Графики радиусов рабочих брусов
        \item Сходимость алгоритма
    \end{itemize}

    \subsection{Нелинейный случай}
    Выбрать систему вида:
    \begin{equation}
        \begin{cases}
            a \cdot x_{1} + b \cdot x_{2} = c \\
            \frac{x_{1}}{x_{2}} = k
        \end{cases}
        \label{eq:nonlinear}
    \end{equation}
    где $ a, b $ - положительные числа, $ c, k $ - положительные интервалы.\newline
    Оценить внешнее множество решений этой системы методом Кравчика.
    Провести вычисления и привести иллюстрации:
    \begin{itemize}
        \item Положения брусов при итерациях
        \item Графики радиусов рабочих брусов
        \item Сходимость алгоритма
    \end{itemize}

    \section{Теория}
    \subsection{Линейный метод Кравчика}
    Рассматриваем ИСЛУА $ \textbf{Ax} = \textbf{b} $.\newline
    Выбираем начальное приближение $ \textbf{x}^{(0)} $ так, чтобы $ \varXi_{uni} \subseteq \textbf{x}^{(0)} $
    и затем итерируем:
    \begin{equation}
        \textbf{x}^{(k+1)} = (\Lambda\textbf{b} + (I - \Lambda\textbf{A})\textbf{x}^{(k)}) \cap \textbf{x}^{(k)}, k = 0, 1, ... 
        \label{eq:Krawczyk}
    \end{equation}
    где $ \Lambda $ - некоторая фиксированная точечная матрица, которая явдяется предобуславливающей матрицей для исходной ИСЛАУ.\newline
    Обычно $ \Lambda $ берут следующим образом.
    \begin{equation}
        \Lambda = (\text{mid}\textbf{A})^{-1}
    \end{equation}
    Если $ \eta = || I - \Lambda\textbf{A} ||_{\infty} \leq 1 $, тогда в качестве начального приближения можно выбрать брус:
    \begin{equation}
        \textbf{x}^{(0)} = ([-\theta, \theta], ..., [-\theta, \theta])^{T}
        \label{eq:boxapprox}
    \end{equation}
    где $ \theta = \frac{|| \Lambda\textbf{b} ||_{\infty}}{1 - \eta} $.\newline
    Предложение.
    Итерационный процесс
    \begin{equation}
        x^{(k+1)} = C(x^{(k)}) + d, k = 0, 1, ...
    \end{equation}
    сходитится, когда $ \rho(|C|) \leq 1 $, где $ |C| $ - матрица составленная из модулей элементов $ C $.
    
    \subsection{Общий метод Кравчика}
    Пусть на брусе $ \textbf{X} \in IR $ задана система $ n $ линейных уравнений с $ n $ неизвестными:
    \begin{equation}
        F(x) = 0
    \end{equation}
    где $ F(x) = \{F_{1}(x), ..., F_{n}(x_{n})\}, x = (x_{1}, ..., x_{n}) $.
    Оператором Кравчика относсительно точки $ \overline{x} $ называется отображение $ K : ID \times R \to IR^{n} $:
    \begin{equation}
        K(\textbf{X}, \overline{x}) = \overline{x} - \Lambda \cdot F(\overline{x}) - (I - \Lambda \cdot \textbf{L}) \cdot (\textbf{X} - \overline{x})
    \end{equation}
    где $ \textbf{L} $ - интервальная матрица Липшица отображения $ F $ на брусе $ \textbf{X} $, $ \Lambda $ - некоторая точечная матрица, выполняющая роль предобуславливателя.\newline
    Тогда итерационный процесс:
    \begin{equation}
        \textbf{X}^{(k+1)} = \textbf{X}^{(k)} \cap K(\textbf{X}^{(k)}, \overline{x}^{(k)})
    \end{equation}
    где $ k = 0, 1, 2,...$, и $ \overline{x}^{(k)} \in \textbf{X}^{(k)} $, сходитсья для некоторого начального бруса $ \textbf{X}^{(0)} $.\newline
    В качестве интервальной матрицы $ \textbf{L} $ можно взять Якобиан $ J(\textbf{X}) $, а в матрицу $ \Lambda = (\text{mid}J(\textbf{X}))^{-1} $

    \section{Реализация}
    Язык программирования: Python. Среда разработки Visual Studio Code.
    \href{https://github.com/kirillkuks/IntervalAnalysis/tree/master/Lab2}{Ссылка на GitHub}

    \section{Результаты}
    \subsection{Линейный случай}
    Рассмотрим систему:
    \begin{equation}
        \begin{cases}
            2 \cdot x_{1} + 3 \cdot x_{2} = [4, 5] \\
            1 \cdot x_{1} - [1, 2] \cdot x_{2} = 0
        \end{cases}
        \label{eq:lineartask}
    \end{equation}
    Матрица $ I - \Lambda A $ имеем вид:
    \begin{equation}
        \begin{pmatrix}
            [0, 0] & [-0.25, 0.25] \\
            [0, 0] & [-0.167, 0.167]
        \end{pmatrix}
    \end{equation}
    Тогда $ \rho(|I - \Lambda A|) \approx 0.1667$, значит итерационный процесс \ref{eq:Krawczyk} сходиться.\newline
    Дадим оценку начального бруса:\newline
    $ \eta = || I - \Lambda A ||_{\infty} = 0.25 < 1 $, тогда справедлива оценка \ref{eq:boxapprox}.
    Вычисляя коэффициент $ \theta $, получим $ \theta \approx 1.667 $.\newline
    Критерий останова: малость изменения бруса,  $ \varepsilon < 10^{-16} $.\newline
    Процесс остановился после 22 итераций в точке : $ \textbf{ x } = ([0.750, 1.50], [0.50, 1.00])^{T} $.\newline
    Приведём соответствующие иллюстрации:

    \plot{LinearBoxes}{Положения брусов при итерациях}{p:linearboxes}
    \plot{LinearRads}{График радиусов рабочих брусов}{p:linearrads}
    \plot{LinearConv}{Сходимость алгоритма}{p:linearconv}

    \subsection{Нелинейный случай}
    Рассмотрим систему с теми же коэффициентами $ a, b $ и интервалами $ c, k $, что и в \ref{eq:lineartask}:
    \begin{equation}
        \begin{cases}
            2 \cdot x_{1} + 3 \cdot x_{2} = [4, 5]\\
            \frac{x_{1}}{x_{2}} = [1, 2]
        \end{cases}
        \label{eq:nonlineartask}
    \end{equation}
    В качестве начального возьмём брус $ X^{(0)} = ([0.25, 4], [0.25, 4])^{T} $.\newline
    Критерий останова: малость изменения бруса,  $ \varepsilon < 10^{-16} $.\newline
    Процесс остановился на 174 итерации в точке: $ \textbf{x} = ([0.250, 2.111], [0.250, 1.412])^{T} $.\newline
    Приведём соответствующие иллюстрации:
    
    \plot{NonlinearBoxes}{Положения брусов при итерациях}{p:nonlinearboxes}
    \plot{NonlinearRads}{График радиусов рабочих брусов}{p:nonlinearrads}
    \plot{NonlinearConv}{Сходимость алгоритма}{p:nonlinearconv}

    \section{Обсуждение}
    Сначала отметим, что системы \ref{eq:lineartask} и \ref{eq:nonlineartask} имеют одинаковое объединённое множество решений, что также видно на рисунках \ref{p:linearboxes}, \ref{p:nonlinearboxes}.
    На рисунках \ref{p:linearboxes} и \ref{p:nonlinearboxes} видно, как с каждой итерацией уменьшается брус, только для линейного случая наблюдается куда более быстрая сходимость.
    Через три итерации радиус бруса почти перестаёт изменяться, а центр бруса почти не перемещается, что также подтверждают рисунки \ref{p:linearrads}, \ref{p:linearconv}.
    В свою очередь для нелинейного случая наблюдается куда более медленная сходимость: для достижения той же точности требуется на порядок больше итераций,
    также на каждой итерации заметно уменьшее радиуса бруса и смещение его центра, рисунки \ref{p:nonlinearrads}, \ref{p:nonlinearconv}.
    При этом в линейном случае брус значительно лучше приближает множество решений системы и со всех сторон почти "касается" множества решений.
    Также стоит отметить, что на рисунке \ref{p:nonlinearboxes} видно, как брус приближается только с двух сторон, и никак не улучшает нижнюю оценку по каждой их координат.

\end{document}
