\documentclass[a4paper,12pt]{article}

\usepackage[hidelinks]{hyperref}
\usepackage{amsmath}
\usepackage{mathtools}
\usepackage{shorttoc}
\usepackage{cmap}
\usepackage[T2A]{fontenc}
\usepackage[utf8]{inputenc}
\usepackage[english, russian]{babel}
\usepackage{xcolor}
\usepackage{graphicx}
\usepackage{float}
\graphicspath{{./images/}}

\definecolor{linkcolor}{HTML}{000000}
\definecolor{urlcolor}{HTML}{0085FF}
\hypersetup{pdfstartview=FitH,  linkcolor=linkcolor,urlcolor=urlcolor, colorlinks=true}

\DeclarePairedDelimiter{\floor}{\lfloor}{\rfloor}

\renewcommand*\contentsname{Содержание}

\newcommand{\plot}[3]{
    \begin{figure}[H]
        \includegraphics[scale=0.9]{#1}
        \caption{#2}
        \label{#3}
    \end{figure}
}

\begin{document}
    \include{Title}

    \tableofcontents
    \newpage

    \section{Постановка задачи}
    Выбрать ИСЛАУ $ 3 \times 2 $
    \begin{equation}
        \begin{cases}
            a \cdot x_{1} + b \cdot x_{2} = c \\
            1 \cdot x_{1} - k \cdot x_{2} = 0 \\
            d \cdot x_{1} + 0 \cdot x_{2} = e
        \end{cases}
        \label{eq:generaltask}
    \end{equation}
    где $ a, b, c, d, e $ - положительные интервалы, $ k $ - положительное число такие, что система \ref{eq:generaltask} несовместна. \newline
    Провести вычисления:
    \begin{itemize}
        \item Максимума распознающего функционала
        \item Достижения разрешимости ИСЛАУ за счёт коррекции правой части
        \item Достижения разрешимости ИСЛАУ за счёт коррекции матрицы
        \item Оценки вариабельности решения 
    \end{itemize} 
    А также для системы исследовать управление положением решения системы \ref{eq:generaltask}:
    \begin{itemize}
        \item Провести исследование управления положением решения за счёт радиусов элементов матрицы в целом
        \item Провести исследование управления положением решения за счёт радиусов элементов матрицы построчно (по уравнениям)
    \end{itemize}
    
    \section{Теория}
    \subsection{Изменение правой части}
    Введём распознающий функционал:
    \begin{equation}
        \text{Tol}(x) = \text{Tol}(x, \textbf{A}, \textbf{b}) = \min_{1 \leq i \leq m}(\text{rad}\textbf{b}_{i} - | \text{mid}\textbf{b}_{i} - \sum_{j = 1}^{n}\textbf{a}_{ij}x_{j} |)
        \label{eq:tol}
    \end{equation}
    который характеризует "дефицит разрешимости". Если $ maxtol = \max_{x}\text{Tol}(x) < 0 $, то ИСЛАУ $ \textbf{A}x=\textbf{b} $ несовместна. \newline
    Для системы $ \textbf{A}x = \textbf{b} + C\textbf{e} $ (где $ \textbf{e} = ([-1, 1], ..., [-1, 1])^{T}, C > 0 $) с уширенной правой частью значение распознающего функционала равно:
    $ \max_{x}\text{Tol}(x, \textbf{A}, \textbf{b} + C\textbf{e} = \max_{x}\text{Tol}(x, \textbf{A}, \textbf{b}) + C $).
    \newline
    Более эффективно расширять разные компоненты правой части по-разному, для этого к правой части будем добавлять:
    $ b + C \nu_{i} \cdot[-1, 1], i = 1, ..., m $, где $ \nu_{i} $ - индивидуальные веса для разных компонент.
    И тогда оперируем модифицированным распознающим функционалом:
    \begin{equation}
        \text{Tol}_{\nu}(x) = \text{min}_{1 \leq i \leq m} (\nu_{i}^{-1}(\text{rad}\textbf{b}_{i} - | \text{mid}\textbf{b}_{i} - \sum_{j = 1}^{n}\textbf{a}_{ij}x_{j} |))
    \end{equation}
    Если системы $ \textbf{A}x = \textbf{b} $ не имела решения, то система с расширенной правой частью $ \textbf{b} + C \nu \cdot \textbf{e} $ при $ C > |maxtol_v| = |\text{Tol}_v(x) | $ становится разрешимой.
    Наиболее важный частный случай $ \nu_{i} = | \textbf{b}_{i} |, i = 1, ..., m $ для ненулевых $ \textbf{b}_{i} $.

    \subsection{Изменение матрицы}
    Пусть ИСЛАУ $ \textbf{A}x = \textbf{b} $ несовместна, $ \tau = \arg\max\text{Tol}(x, \textbf{A}, \textbf{b}) $ и $ \text{Tol}(\tau, \textbf{A}, \textbf{b}) = T < 0 $.
    При условии, что у ИСЛАУ:
    \begin{itemize}
        \item все компоненты вектора правой части $ \textbf{b} $ являются не невырожденными интервалами ($ \text{rad}\textbf{b}_{i} > 0, \forall i = 1, ..., m $)
        \item в каждой строке матрицы $ \textbf{A} $ существуют элементы с ненулевой щириной ($ \sum_{j = 1}^{n}\text{rad}\textbf{a}_{ij} > 0, \forall i = 1, ..., m $)
    \end{itemize}
    можно добиться совместности системы следующим образом. \newline
    Используя величины $ | \tau_{i} | $ в качестве весовых множителей, посчитаем \newline
    $ \min_{1 \leq i \leq m}(\sum_{j = 1}^{n}| \tau_{i} | \text{rad}\textbf{a}_{ij}) = \Delta $.
    Построим интервальную матрицу $ m \times n $ $ E = (\textbf{e}_{ij}) $, где $ \textbf{e}_{ij} = [-e_{ij}, e_{ij}] $ такие, что $ \sum_{j = 1}^{n}e_{ij}\tau_{j} = K, i = 1, ..., m $.
    $ K $ - некоторая положительная константа $ 0 < K \leq \Delta $, и $ \textbf{a}_{ij} \geq e_{ij} \geq \forall i, j $. \newline
    Тогда ИСЛАУ с тем же вектором правых частей $ \textbf{b} $ и матрицей $ \textbf{A} \ominus \textbf{E} $ является "менее неразрешимой". \newline
    $ \max_{x}\text{Tol}(x, \textbf{A} \ominus \textbf{E}, \textbf{b}) \geq K + \text{Tol}(\tau, \textbf{A}, \textbf{b}) = K + T $.

    \subsection{Оценивание вариабельности решения}
    Для интервально заданных величин имеют место две оценки: \newpage
    Оценка абсолютной вариабельности: 
    \begin{equation}
        \text{ive}(\textbf{A}, \textbf{b}) = \min_{A \in \textbf{A}} \text{cond}A \cdot || \arg\max \text{Tol} || \cdot \frac{\max \text{Tol}}{|| \textbf{b} ||}
    \end{equation}
    \noindent
    Оценка относительной вариабельности:
    \begin{equation}
        \text{rve}(\textbf{A}, \textbf{b}) = \min_{A \in \textbf{A}} \text{cond}A \cdot \max\text{Tol}
    \end{equation}

    \section{Реализация}
    Язык программирования: Python. Среда разработки: Visual Studio Code.
    \href{https://github.com/kirillkuks/IntervalAnalysis/tree/master/Lab3}{Ссылка на GitHub}

    \section{Результаты}
    Рассмотрим ИСЛАУ:
    \begin{equation}
        \begin{cases}
            [0.5, 1.5] \cdot x_{1} + [1.5, 2.5] \cdot x_{2} = [3, 5] \\
            1 \cdot x_{1} - 3 \cdot x_{2} = 0 \\
            [-0.5, 0.5] \cdot x_{1} + 0 \cdot x_{2} = [-1, 1]
        \end{cases}
        \label{eq:task}
    \end{equation}
    \noindent
    Проверим несовместность системы \ref{eq:task}. $ argmax = [2.182, 0.909], maxtol = -0.545 < 0 $, значит система не совместна.
    
    \subsection{Коррекция правой части}
    Так как в правой части ИСЛАУ \ref{eq:task} есть нуль, то нельзя воспользоваться модифицированным распознающим функционалом с $ \nu_{i} = | \textbf{b}_{i} | $.
    Поэтому будем использовать распознающий функционал \ref{eq:tol}. \newline
    После коррекции правой части получили следующий вектор правых частей:
    \begin{equation}
        \textbf{b}^{'} = ([2.456, 5.545], [-0.545, 0.545], [-1.545, 1.545])^{T}
        \label{eq:newb}
    \end{equation}
    И система $ \textbf{A}x = \textbf{b}^{'} $ совместна: $ argmax = (2.182, 0.909), maxtol = 3.5 \cdot 10^{-8} $. \newline
    Оценки вариабельности: \newline
    $ \text{ive} = 3.7 \cdot 10^{-8}, \text{rve} = 9.2 \cdot 10^{-8} $.

    \subsection{Коррекция матрицы}
    Так как для ИЛАУЮ \ref{eq:task} не выполнены условия из пункта $ 2.2 $, то метод из $ 2.2 $ неприменим.
    Поэтому поступим следующим образом: на каждой итреции будем уменьшать радиусы интревалов в матрице в $ 2 $ раза до тех пор, пока максимальное значение распознающего функционала не станет положительным (или хотя бы близким к нулю).
    После сужения элементов матрицы получили следующую матрицу:
    \begin{equation}
        \textbf{A}^{'} = \begin{pmatrix}
            [0.9843, 1.0156] & [1.9843, 2.0156] \\
            1 & -3 \\
            [-0.0156, 0.0156] & 0
        \end{pmatrix}
    \end{equation}
    И система $ \textbf{A}^{'}x = \textbf{b} $ совместна $ argmax = (1.823, 0.608), maxtol = 10^{-11} $. \newline
    Оценки вариабельности: \newline
    $ \text{ive} = 9.9 \cdot 10^{-12}, \text{rve} = 2.6 \cdot 10^{-11} $.

    \subsection{Управление аргументом решения ИСЛАУ}
    \textbf{Управления положением решения за счёт радиусов элементов матрицы в целом} \newline
    В каждой строке на каждой итерации радиусы интервалов уменьшаются в $ 2 $ раза.
    \plot{MatrixCorrection}{Положение решения за счёт радиусов элементов матрицы в целом}{p:matrixcorrection}

    \noindent
    \textbf{Управления положением решения за счёт радиусов элементов матрицы построчно} \newline
    На каждой итерации уменьшаются радиусы только в одной строке в $ 2 $ раза.
    \plot{MatrixCorrectionByLine0}{Положение за счёт радиусов элементов матрицы в первой строке}{p:matrixcorrectionbyline0}
    \plot{MatrixCorrectionByLine2}{Положение за счёт радиусов элементов матрицы в третьей строке}{p:matrixcorrectionbyline2}

    \section{Обсуждения}
    Из результатов видно, что в случае изменения правой части точка максимума распознающего функционала не меняется после расширения интервалов в векторе правых частей.
    А в случае с изменением матрицы значение максимума изменяется, при чем все значения максимума, кроме начального, лежат на одной прямой, что видно на рисунке \ref{p:matrixcorrection}.
    Также стоит отметить, что коррекция матрицы даёт меньшие значения для оценок вариабельности $ \text{ive}, \text{rve} $. \newline
    Рисунки \ref{p:matrixcorrection}, \ref{p:matrixcorrectionbyline0} сильно похожи, положение точки максимума распознающего функционала отличаются только на первых итерациях при сужении интервалов, и то не сильно, а сами точки лежат на одно прямой, соответствующей второму уравнению системы \ref{eq:task}.
    Отсюда можно сделать вывод, что для системы \ref{eq:task} наибольший вклад в изменение точки максимума распознающего функционала вносит первое уравнение системы.
    Второе уравнение не может вносить изменения, так как не содержит интервальных величин. А на рисунке \ref{p:matrixcorrectionbyline2} видно, что сужение интервала в только в третьем уравнении никак не влияет на положение точки максимума распознающего функционала, что подтверждает предыдущий вывод. 

\end{document}