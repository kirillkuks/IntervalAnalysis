\documentclass[a4paper,12pt]{article}

\usepackage[hidelinks]{hyperref}
\usepackage{amsmath}
\usepackage{mathtools}
\usepackage{shorttoc}
\usepackage{cmap}
\usepackage[T2A]{fontenc}
\usepackage[utf8]{inputenc}
\usepackage[english, russian]{babel}
\usepackage{xcolor}
\usepackage{graphicx}
\usepackage{float}
\graphicspath{{./images/}}

\definecolor{linkcolor}{HTML}{000000}
\definecolor{urlcolor}{HTML}{0085FF}
\hypersetup{pdfstartview=FitH,  linkcolor=linkcolor,urlcolor=urlcolor, colorlinks=true}

\DeclarePairedDelimiter{\floor}{\lfloor}{\rfloor}

\renewcommand*\contentsname{Содержание}

\newcommand{\plot}[3]{
    \begin{figure}[H]
        \includegraphics[scale=0.7]{#1}
        \caption{#2}
        \label{#3}
    \end{figure}
}

\begin{document}
    \include{Title}
    \newpage
    
    \tableofcontents
    \newpage

    \section{Постановка задачи}
    \subsection{Получение решения по теореме Зюзина}
    Выбрать ИСЛАУ $ 2 \times 2 $. Построить итерационную схему с разложением матрицы на диагональную и недиагональную части.\newline
    Провести вычисления и привести иллюстрации:
    \begin{itemize}
        \item Брусов итерационного процесса
        \item Радиусов решения в зависимости от номера итерации
    \end{itemize}

    \subsection{Получение формального решения ИСЛАУ субдифференциальным методом Ньютона}
    Для ИСЛАУ
    \begin{equation}
        \begin{pmatrix}
            [3, 4] & [5, 6] \\
            [-1, 1] & [-3, 1] \\
        \end{pmatrix}
        \cdot x = 
        \begin{pmatrix}
            [-3, 3] \\
            [-1, 2]
        \end{pmatrix}
        \label{eq:subdifftask1}
    \end{equation}
    построить итерационную схему субдифференциального метода Ньютона.\newline
    Провести вычисления и привести иллюстрации:
    \begin{itemize}
        \item Брусов итерационного процесса
        \item Сравнить результаты с решением ИСЛАУ
        \begin{equation}
            \begin{pmatrix}
                [3, 4] & [5, 6] \\
                [-1, 1] & [-3, 1] \\
            \end{pmatrix}
            \cdot x = 
            \begin{pmatrix}
                [-3, 4] \\
                [-1, 2]
            \end{pmatrix}
        \end{equation}
        \label{eq:subdifftask2}
    \end{itemize}

    \section{Теория}
    \subsection{Теорема Зюзина}
    Говорят, что квадартная интервальная $ n \times n $ матрица $ \textbf{A} $ иммет диагональное преобладание, если для любого $ i = 1, 2, ..., n $, если
    \begin{equation}
        \langle \textbf{a}_{ii} \rangle > \sum_{j \neq i} | \textbf{a}_{ij} |
    \end{equation}
    $ \langle \textbf{a} \rangle $ - мигнитуда интервала. \newline
    
    \noindent\textbf{Теорема Зюзина} \newline
    Для ИСЛАУ
    \begin{equation}
        \textbf{C}x = \textbf{d}
    \end{equation}
    где $ \textbf{C} \in KR^{n \times n}, \textbf{d} \in KR^{n} $. Правильная проекция матрицы $ \textbf{C} $ имеет диагональное преобладание.
    Тогда решение системы существует и единственно.
    Пусть $ \textbf{D} = diag\{ \textbf{c}_{11}, \textbf{c}_{22}, ..., \textbf{c}_{nn} \}, \textbf{E} $ - матрица, полученная из $ \textbf{C} $ занулением диагональных элементов.
    Тогда для некоторого $ x^{0} $ итерационный процесс:
    \begin{equation}
        x^{k + 1} = \text{inv}\textbf{D}(\textbf{d} \ominus \textbf{Ex}^{(k)}), k = 0, 1, ...
        \label{eq:iterzyuzin}
    \end{equation}
    в силу диагонального преобладания $ \textbf{C} $ будет сходиться к единственной неподвижной точке.

    \subsection{Субдифференциальный метод Ньютона}
    Рассматриваем ИСЛАУ:
    \begin{equation}
        \textbf{C}x = \textbf{d}
        \label{eq:subdiffgeneraltask}
    \end{equation}
    Отображение $ si: KR^{n} \to R^{2n} $ вида
    \begin{equation}
        (\textbf{x}_{1}, \textbf{x}_{2}, ..., \textbf{x}_{n}) \to 
        (\underline{x}_{1}, \underline{x}_{2}, ..., \underline{x}_{n}, \overline{x}_{1}, \overline{x}_{2}, ..., \overline{x}_{n})
    \end{equation}
    называется простейшим погружением. \newline
    Для решения индуцированных уравнения $ G(y) = 0 $ такого что
    \begin{equation}
        G(y) = \text{si}(\textbf{C}\text{si}^{-1}(\textbf{y})) - \text{si}(\textbf{d})
    \end{equation}
    в $ R^{2n} $ развит субдифференциальный метод Ньютона: \newline
    Выбираем некоторое начальное приближение $ x^{0} \in R^{2n} $. Если $ (k - 1) $ приближение $ x^{(k - 1)} \in R^{2n}, k = 1, 2, ... $ уже найдено,
    то вычисляем какой-нибудь субградиент $ D^{(k - 1)} $ отображение $ G $ в точке $ x^{(k - 1)} $ и полагаем
    \begin{equation}
        x^{(k)} = x^{(k - 1)} - \tau(D^{(k - 1)})^{-1}G(x^{(k - 1)})
        \label{eq:itersubdiff}
    \end{equation}
    где $ \tau \in [0, 1] $ - некотарая константа.
    Начальное приближение можно найти из решения "средней" системы.
    \begin{equation}
        (\text{mid}\textbf{C})^{'} \cdot x^{(0)} = \text{si}\textbf{d}
    \end{equation}
    где через $ ' $ обозначена точечная матрица вида:
    \begin{equation}
        A = 
        \begin{pmatrix}
            A^{+} & -A^{-} \\
            -A^{-} & A^{+}    
        \end{pmatrix}
    \end{equation}
 
    \section{Реализация}
    Язык программирования: Python. Среда разработки: Visual Studio Code.
    \href{https://github.com/kirillkuks/IntervalAnalysis/tree/master/Lab4}{Ссылка на GitHub}.

    \section{Результаты}
    \subsection{Получение решения по теореме Зюзина}
    Возьмём матрицу
    \begin{equation}
        \textbf{C} = 
        \begin{pmatrix}
            [5, 6] & [3, 4] \\
            [-1, 1] & [2, 3]
        \end{pmatrix}
    \end{equation}
    и вектор $ \textbf{x} = ([1, 2], [2, 4])^{T} $ и построим вектор правых частей:
    \begin{equation}
        \textbf{b} = \textbf{C} \cdot x =
        \begin{pmatrix}
            [11, 28] \\
            [2, 14]
        \end{pmatrix}
    \end{equation}

    \noindent Будем рассмотривать систему:
    \begin{equation}
        \begin{cases}
            [5, 6] \cdot x_{1} + [3, 4] \cdot x_{2} = [11, 28] \\
            [-1, 1] \cdot x_{1} + [2, 3] \cdot x_{2} = [2, 14]
        \end{cases}
        \label{eq:zyuzintask}
    \end{equation}
    В качестве начального приближения возьмём точку $ \textbf{x}^{(0)} = ([-10, 10], [-10, 10])^{T} $
    Видно, что интервальная матрица $ \textbf{C} $ имеет диагональное преобладание.
    Значит для ИСЛАУ \ref{eq:zyuzintask} справедлива теорема Зюзина. \newline
    Критерий останова итерационного процесса \ref{eq:iterzyuzin} - малость изменения бруса на текущей итерации относительно бруса на предыдущей итерации: $ \varepsilon < 10^{-16} $. \newline

    \noindent Процесс остановился после $ 52 $ итераций в точке $ \textbf{x} = ([1.0, 2.0], [2.0, 4.0])^{T} $. \newline
    Приведём соответствующие иллюстрации:
    \plot{ZyuzinBoxes}{Положения брусов при итерациях}{p:zuyzinboxes}
    \plot{ZyuzinRadiuses}{График радиусов брусов в зависимости от номера итерации}{p:zyuzinradiuses}

    \subsection{Получение формального решения ИСЛАУ субдифференциальным методом Ньютона}
    Сначала рассмотрим решение системы \ref{eq:subdifftask1}. \newline
    Критерий останова итерационного процесса \ref{eq:itersubdiff} - малость изменения бруса на текущей итерации относительно бруса на предыдущей итерации: $ \varepsilon < 10^{-16} $.
    Параметр $ \tau = 1 $.

    \noindent Процесс остановился после $ 4 $ итераций в точке $ \textbf{x} = ([0.0, 0.5], [-0.5, 0.167])^{T} $. \newline
    Приведём соответствующие иллюстрации.
    \plot{SubdifferentialNewtonTask1Boxes}{Положения брусов при итерациях}{p:subdifftask1boxes}
    \plot{SubdifferentialNewtonTask1Radiuses}{График радиусов брусов в зависимости от номера итерации}{p:subdifftask2radiuses}

    \noindent
    Теперь рассмотрим решение системы \ref{eq:subdifftask2}. \newline
    Итерационный процесс не сходится, а через $ 8 $ итерации уходит в цикл длиной $ 4 $ точки. 
    Параметр $ \tau = 1 $.
    Соответствующие иллюстрации для первых $ 50 $ итераций.
    \plot{SubdifferentialNewtonTask2Tau1Boxes}{Положения брусов при итерациях, $ \tau = 1 $}{p:subdifftask2tau1boxes}
    \plot{SubdifferentialNewtonTask2Tau1Radiuses}{График радиусов брусов в зависимости от номера итерации, $ \tau = 1 $}{p:subdifftask2tau1radiuses}

    \noindent
    Уменьшим параметр $ \tau = 0.1 $.
    Соответствующие иллюстрации для первых 100 итераций.
    \plot{SubdifferentialNewtonTask2Tau01Boxes}{Положения брусов при итерациях, $ \tau = 0.1 $}{p:subdifftask2tau01boxes}
    \plot{SubdifferentialNewtonTask2Tau01Radiuses}{График радиусов брусов в зависимости от номера итерации, $ \tau = 0.1 $}{p:subdifftask2tau01radiuses}
    Итерационный процесс также не сходится, как и в случае с $ \tau = 1 $, а ходит по циклу той длины $ 5 $.

    \section{Обсуждения}
    Из результатов решения системы \ref{eq:zyuzintask} видно, что итерационная схема с разложением матрицы на диагональную и недиагональную для системы, удовлетворяющей условию теормы Зюзина, сходиться.
    На рисунках \ref{p:zuyzinboxes} - \ref{p:zyuzinradiuses} можно заметить, что до четвёртой итерации радиус бруса монотонно убывает и, достигнув минимального значения, 
    которое сильно меньше радиуса бруса решения, на четвёртой итерации, затем постопенно начинает сходиться к решению системы. \newline

    \noindent
    Из результатов решения систем \ref{eq:subdifftask1}, \ref{eq:subdifftask2} видно, что у субдифференциального метода Ньютона могут возникнуть проблемы со сходимостью.
    Итерационный процесс \ref{eq:itersubdiff} для системы \ref{eq:subdifftask1} сходится достаточно быстро.
    В свою очередь для системы \ref{eq:subdifftask2} процесс не сходится, а зацикливается.
    В таком случае подбор параметра $ \tau $ может улучшить ситуацию:
    \plot{SubdifferentialNewtonTask2Tau1Cyrcle}{Цикл брусьев при итерациях, система \ref{eq:subdifftask2}, $ \tau = 1 $}{p:subdifftask2tau1cyrcle}
    \plot{SubdifferentialNewtonTask2Tau01Cyrcle}{Цикл брусьев при итерациях, система \ref{eq:subdifftask2}, $ \tau = 0.1 $}{p:subdifftask2tau01cyrcle}
    На рисунках \ref{p:subdifftask2tau1cyrcle} - \ref{p:subdifftask2tau01cyrcle} видно, что при значении $ \tau = 0.1 $ брусья в цикле изменяются на каждой итерации меньше, чем при $ \tau = 1 $.
    Хотя радиус брусьев почти не менятеся при разных значениях $ \tau $, что видно на рисунках \ref{p:subdifftask2tau1radiuses}, \ref{p:subdifftask2tau01radiuses}.
    Также стоит отметить,что при $ \tau = 0.1 $ средний брус в цикле $ \textbf{x} = ([-0.15, 0.8], [-0.4, 0.13]) $ достаточно близко к решению системы \ref{eq:subdifftask2}. \newline
\end{document}
